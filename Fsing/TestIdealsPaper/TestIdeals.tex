\documentclass[11pt]{amsart}
\usepackage{calc,amssymb,amsthm,amsmath,fullpage    }
%\usepackage{mathtools}
\RequirePackage[dvipsnames,usenames]{xcolor}
\usepackage{hyperref}
\hypersetup{
bookmarks,
bookmarksdepth=3,
bookmarksopen,
bookmarksnumbered,
pdfstartview=FitH,
colorlinks,backref,hyperindex,
linkcolor=Sepia,
anchorcolor=BurntOrange,
citecolor=MidnightBlue,
citecolor=OliveGreen,
filecolor=BlueViolet,
menucolor=Yellow,
urlcolor=OliveGreen
}
\usepackage{alltt}
\usepackage{multicol}
%\usepackage{etex}
\usepackage{xspace}
\usepackage{rotating}
\interfootnotelinepenalty=100000

\usepackage{mabliautoref}
\usepackage{colonequals}
\frenchspacing
\input{kmacros3.sty}
\usepackage{stmaryrd}

\usepackage{verbatim}
\usepackage{enumerate}

\DeclareMathOperator{\HH}{H}
%\DeclareMathOperator{\Image}{image}

\begin{document}
\title{{TestIdeals} package for \emph{Macaulay2}}
\author{Karl Schwede}
\date{\today}
\address{Department of Mathematics, University of Utah, 155 S 1400 E Room 233, Salt Lake City, UT, 84112}
\email{schwede@math.utah.edu}

\begin{abstract}
  This note describes a \emph{Macaulay2} package for computations in commutative rings prime related to $p^{-e}$-linear and $p^{e}$-linear  maps,
  singularities defined in terms of these maps,  various test ideals and modules, and ideals compatible with a given $p^{-e}$-linear map.
\end{abstract}


\subjclass[2010]{13A35}

\keywords{Macaulay2}

%\thanks{The first named author was supported in part by the NSF FRG Grant DMS \#1265261, NSF CAREER Grant DMS \#1252860 and a Sloan Fellowship.}
%\thanks{The second named author was supported in part by the NSF CAREER Grant DMS \#1252860.}
\maketitle

\section{Introduction}

This paper constructive methods for computing various objects related to commutative rings of prime characteristic $p$.
Such a ring $R$ comes equipped with a built-in endomorphism, namely the Frobenius endomorphism $f:R \rightarrow R$ which is the basis for many constructions and definitions
which affords a handle on many problems which is not otherwise available. Two notable examples of the use of the Frobenius endomorphism are the theory of tight closure {\hfill\large\color{red} [Add references]}\\
and the resulting theory of test ideals. {\hfill\large\color{red} [Add references]}\\

{\large\color{red} [Add history of the package]}

\subsection*{Acknowledgements}

We thank ??? for useful conversations and comments on the development of this package.

\section{Frobenius powers and Frobenius roots}\label{Section: Frobenius powers and Frobenius roots}

%%% Throughout  this paper $R$ will denote a polynomial ring over a field $\mathbb{K}$ of prime characteristic $p$.

Let $S$ denote any commutative ring of prime characteristic $p$.

\begin{definition}
For any ideal $I\subseteq S$ and any integer $e\geq I$, we define the \emph{$e$th Frobenius power of $I$} to be the ideal denoted $I^{[p^e]}$ which is
generated by all $p^e$th powers of elements in $I$.
\end{definition}

It is easy to see that, if $I$ is generated by $g_1, \dots, g_\ell$, $I^{[p^e]}$ is generated by $g_1^{p^e}, \dots, g_\ell^{p^e}$.


\begin{definition}
For any ideal $I\subseteq S$ and any integer $e\geq I$, we define the \emph{$e$th Frobenius root of $I$} to be the ideal denoted $I^{[1/p^{e}]}$ which is
the smallest ideal $J$ such that $I\subseteq J^{[p^e]}$, if such ideal exists.
\end{definition}

$e$th Frobenius roots exist in polynomial rings (cf.~\cite[\S 2]{BlickleMustataSmithDiscretenessAndRationalityOfFThresholds}) and in power series rings
(cf.~\cite[\S 5]{KatzmanParameterTestIdealOfCMRings}).


\begin{verbatim}
i2 :      R=ZZ/5[x,y,z]
i3 :      I=ideal(x^6*y*z+x^2*y^12*z^3+x*y*z^18)
                18    2 12 3    6
o3 = ideal(x*y*z   + x y  z  + x y*z)
o3 : Ideal of R
i4 :      frobeniusPower(1/5,I)
                2   3
o4 = ideal (x, y , z )
\end{verbatim}

We can extend the definition of Frobenius powers as follows
\begin{definition}{\hfill\large\color{red} [Add references]}\\
Let  $I\subseteq S$ be an ideal.
\begin{enumerate}
 \item[(a)] If $a$ is a positive integer with base-$p$ expansion  $a=a_0 + a_1 p +  \dots + a_r p^r$, we define
 $I^{[n]}=I^{a_0} \left(I^{a_1}\right)^{[p]} \dots  (I^{a_r})^{[p^r]}$. %\left( I^{a_r}\right)^{[p^r]}$ .
 \item[(b)] If $t$ is a non-negative rational number of the form $t = a/p^e$, we define  $I^{[t]} = (I^{[a]})^{[1/p^e]}.$
 \item[(c)] If $t$ is any non-negative rational number, and $\{a_n/p^{e_n}\}_{n\geq 1}$ is a sequence of rational numbers converging to $t$ from above, we define $I^{[t]}$
 to be the stable value of increasing chain of ideals $\{I^{[a_n/p^{e_n}]}\}_{n\geq 1}$.

\end{enumerate}
\end{definition}


\begin{verbatim}
i5 :      frobeniusPower(1/2, ideal(y^2-x^3))
o5 = ideal 1
o5 : Ideal of R
i6 :      frobeniusPower(5/6, ideal(y^2-x^3))
o6 = ideal (y, x)
o6 : Ideal of R
\end{verbatim}



\section{$p^{-e}$- and $p^{e}$-linear maps}\label{Section: p-linear maps}

\begin{definition}%{\hfill\large\color{red} [Add references]}\\
Let  $M$ be an $S$-module and $e$ a non-negative integer.
\begin{enumerate}
 \item[(a)] A $p^{-e}$-linear map $\phi:M \rightarrow M$ is an additive map such that
 $\phi(s^{p^e} m)= s\phi(m)$ for all $s\in S$ and $m\in M$.
 \item[(b)] A $p^{e}$-linear map $\psi:M \rightarrow M$ is an additive map such that
 $\phi(s m)= s^{p^e}\phi(m)$ for all $s\in S$ and $m\in M$.
\end{enumerate}
\end{definition}


The following two examples describe two prototypical
$p^{-e}$- and $p^{e}$-linear maps.
\begin{example}
For any $S$-module $M$, we can construct a new $S$-module $F_*^e M$ with elements $\{ F_*^e m \,|\,m\in M\}$ by defining
$F_*^e m_1 + F_*^e m_2 = F_*^e (m_1 +  m_2)$ for all $m_1, m_2 \in M$ and
$s F_*^e m= F_*^e s^{p^e}$ for all $m\in M$ and $s\in S$.

Consider any $\phi\in \Hom_S(F_*^e M, M)$: if we identify $F_*^e M$ with $M$ we can interpret $\phi$ as a $p^{-e}$-linear map.
\end{example}

\begin{example}
The $e$th Frobenius map $f:S \rightarrow S$  raising elements to their $p^e$th power is clearly $p^{e}$-linear.
Furthermore, any ideal $I\subseteq S$, $f$ induces an $p^{e}$-linear map $\HH_I^k (S) \rightarrow \HH_I^k (S)$.
\end{example}

Let $R$ be a polynomial ring with irrelevant ideal $\mathfrak{m}$ and let $g\in R\setminus \{0\}$.
Let $E=E_{R\mathfrak{m}}(R_{\mathfrak{m}}/\mathfrak{m})$ denote the injective hull of $R_{\mathfrak{m}}/\mathfrak{m}$.

The Frobenius map on $R$ induces a Frobenius map $S$ and on
$\HH^{\dim R-1}_{\mathfrak{m}} (S)=E_S(S/\mathfrak{m}S)=\Ann_E g$ and the kernel of this map is given by
$\Ann_E (g^{p-1}R)^{[1/p]}$ (cf.~\cite[\S 5]{KatzmanParameterTestIdealOfCMRings}).

\begin{verbatim}
i2 :      R=ZZ/5[x,y,z]
i3 :      g=x^3+y^3+z^3
i4 :      u=g^(5-1)
i5 :      frobeniusPower(1/5,ideal(u))
o5 = ideal (z, y, x)
o5 : Ideal of R

i6 :      R=ZZ/7[x,y,z]
i7 :      g=x^3+y^3+z^3
i8 :      u=g^(7-1)
i9 :      frobeniusPower(1/7,ideal(u))
o9 = ideal 1
\end{verbatim}

Thus we see that the induced $p^{e}$-linear map on $\HH_{(x,y,z)}^{2} \left( \mathbb{K}[x,y,z]/(x^3+y^3+z^3) \right)$ is injective
when the characteristic of $\mathbb{K}$ is $7$ and non-injective when the characteristic is $5$.



\section{$F$-singularities}

\section{Test ideals}

In this section, we explain how to compute parameter test modules, parameter test ideals, test ideals and HSLG-modules\footnote{HSLG-modules can be used to give a scheme structure to the $F$-injective or $F$-pure locus.}.

\subsection{Parameter test modules}

Given an $F$-finite reduced ring $R$, the Frobenius map $R \to R^{1/p^e}$ is dual to $T : \omega_{R^{1/p^e}} \to \omega_R$.  As before in \autoref{}, we can represent $\omega_R \subseteq R$ as an ideal, we can write $R = S/I$, and so we can find a $u \in S^{1/p^e}$ representing the map $\omega_{R^{1/p^e}} \to \omega_R$ \cite{}.  See also \autoref{} below.

The parameter test submodule is the smallest submodule $M \subseteq \omega_R$ (and hence ideal of $R$ since $M \subseteq R$) that agrees generically with $\omega_R$ and that satisfies
\[
T (M^{1/p^e}) \subseteq M.
\]
From within Macualay2, we can compute this using the {\tt testModule} command as follows.
\begin{verbatim}
i3 : R = ZZ/5[x,y,z]/ideal(x^4+y^4+z^4);
i4 : N = testModule(R);
i5 : N#0
             2             2        2
o5 = ideal (z , y*z, x*z, y , x*y, x )
o5 : Ideal of R
i6 : N#1
o6 = ideal 1
o6 : Ideal of R
\end{verbatim}
The output of {\tt testModule} is a list with three items.  The first entry is the test module, the second is the canonical module (as an ideal) that it lives inside, and the third is the element $u$ described above (not listed here, since it is rather complicated).  Note since this ring is Gorenstein, the canonical module is simply represented as the unit ideal.

Here is another example where the ring is not Gorenstein.
\begin{verbatim}
i2 : R = ZZ/5[x,y,z]/ideal(y*z, x*z, x*y);
i3 : N = testModule(R);
i4 : N#0
             2   2   2
o4 = ideal (z , y , x )
o4 : Ideal of R
i5 : N#1
o5 = ideal (y - z, x + z)
o5 : Ideal of R
\end{verbatim}

We briefly explain how this is computed.  First we find a \emph{test element}.
\begin{remark}[Computation of test elements]
\label{rem.ComputationOfTestElements}
Roughly, we recall that an element of the Jacobian ideal that is not contained in any minimal prime is a test element \cite{}.  We compute test elements by computing random partial derivatives (and linear combinations thereof) until we find an element that doesn't vanish at all the minimal primes.  This method is much faster than computing the entire Jacobian ideal.  If you know your ring is a domain, you can use the {\tt AssumeDomain} flag to speed this up further.
\end{remark}

After the test element $c$ has been identified, we pullback the ideal $\omega_R$ to an ideal $J = S$.  Then we compute the following ascending sequence of ideals
\begin{equation}
\label{eq.AscendIdealExplanation}
\begin{array}{rll}
J_1 := & = J + (c J)^{[1/p]} & = {\tt J + frobeniusRoot(1, c*J)}\\
J_2 := & = J + (c J)^{[1/p]} + (c^{1+p} J)^{[1/p^2]} & = {\tt J + frobeniusRoot(1, c*J_1)}\\
J_3 := & = J + (c J)^{[1/p]} + (c^{1+p} J)^{[1/p^2]} + (c^{1+p+p^2} J)^{[1/p^3]} & = {\tt J + frobeniusRoot(1, c*J_2)}.\\
\dots
\end{array}
\end{equation}
As soon as this ascending sequence of ideals stabilizes, we are done.
In fact, because this strategy is used in several contexts, you can call it directly for a chosen ideal $J$ and $u$ with the function {\tt ascendIdeal} (this is done for test ideals below).

We can also compute parameter test modules of pairs $\tau(\omega_R, f^{t})$ with $t \in \mathbb{Q}_{\geq 0}$.  This is done by modifying the element $u$ when the denominator of $t$ is not divisible by $p$.  When $t$ has $p$ in the denominator, we rely on the fact (see \cite{}) that
\[
\begin{array}{rcl}
T(\tau(\omega_R, f^a)) & = & \tau(\omega_R, f^{a/p})\\
{\tt frobeniusRoot(1, u*I_1)} & {\tt =} & {\tt I_2}
\end{array}
\]
Where the second line roughly explains how this is accomplished internally, here ${\tt I_1}$ is $\tau(\omega_R, f^a)$ pulled back to $S$ and likewise $I_2$ defines $\tau(\omega_R, f^a)$ modulo the defining ideal of $R$.

\begin{remark}[Optimizations in {\tt ascendIdeal} and other {\tt testModule} computations]
Throughout the computations described above, we very frequently use the following fact:
\[
(f^p \cdot J)^{[1/p]} = f \cdot (J^{[1/p]}).
\]
To access this enhancement, one should try to pass functions like {\tt ascendIdeal} and {\tt frobeniusRoot} the elements and their exponents (see the documentation).  In particular, do not multiply out things like $f^n \cdot J$ if you will be applying {\tt frobeniusRoot}.  
\end{remark}

\subsection{Parameter test ideals}

The parameter test ideal is simply the annihilator of $\omega_R/\tau(\omega_R)$.  Or in other words, it is
\[
\tau(\omega_R) : \omega_R.
\]
This can also be described as 
\[
\bigcap_{I} (I^* : I)
\]
where $I$ varies over ideals defined by a partial system of parameters and $I^*$ denotes its tight closure.  This latter description is not computable however.  

\begin{verbatim}
todo (Moty will insert a good example)
\end{verbatim}

\subsection{Test ideals}

For an $F$-finite reduced ring $R = S/I$ where $S$ is a regular ring, the test ideal of $R$ is the smallest ideal $J$ in $R$, not contained in any minimal prime, such that for all $e > 0$ and all
\[
\phi \in \Hom_R(R^{1/p^e}, R) \text{ we have } \phi(J^{1/p^e}) \subseteq J.
\]
In the case that $R$ is Gorenstein, there is a map $\Phi_e \in \Hom_R(R^{1/p^e}, R)$ which generates the $\Hom$-set as an $R^{1/p^e}$-module.  In fact this $\Phi_e$ is identified with the map $T$ above based on the identification $\omega_R \cong R$, \cite{}.  More generally, if $R$ is $\bQ$-Gorenstein with index not divisible by $p$, then for at least sufficiently divisible $e > 0$, such a generating $\Phi_e$ still exists \cite{}.  

If such a $\Phi_e$ exists, it can be identified as the generator of the module $(I^{[p^e]} : I) / I^{[p^e]}$ by Fedder's Lemma, \cite{}.  Hence we can find a corresponding\footnote{This is done via the function {\tt QGorensteinGenerator}. } $u \in I^{[p^e]} : I$.  In this case, if $c \in S$ is the pre-image of a test element of $R$, then setting $I_0 = cR$, it follows that
\[
\tau(R) = {\tt ascendIdeal(e, u, I_0)}.
\]
Recall that {\tt ascendIdeal} is explained above in \autoref{eq.AscendIdealExplanation}.  The $e$ here means all $e$th roots are taken as multiples of $e$.  See \cite{}.  

Here is an example (a $\bZ/3\bZ$-quotient where $3 | (7-1)$), where exactly this logic occurs.  
\begin{verbatim}
i3 : T = ZZ/7[x,y];
i4 : S = ZZ/7[a,b,c,d];
i5 : f = map(T, S, {x^3, x^2*y, x*y^2, y^3});
i6 : I = ker f;
i7 : R= S/I;
i8 : testIdeal(R)
o8 = ideal 1
o8 : Ideal of R
\end{verbatim}
However, the term $u$ constructed above can be quite involved if $e > 1$ (which happens exactly when $(p -1)K_R$ is not Cartier).  For instance, even in the above example:
\begin{verbatim}
i9 : toString(QGorensteinGenerator(1, R))
o9 = a^2*b^6*c^12+a^3*b^4*c^13+a^3*b^5*c^11*d+a^4*b^3*c^12*d+a^5*b*c^13*d+b^12
      *c^6*d^2+a^3*b^6*c^9*d^2+a^4*b^4*c^10*d^2+a^5*b^2*c^11*d^2+a^6*c^12*d^2+b
      ^13*c^4*d^3+a*b^11*c^5*d^3+a^2*b^9*c^6*d^3+a^4*b^5*c^8*d^3+a^5*b^3*c^9*d^
      3+a^6*b*c^10*d^3+a*b^12*c^3*d^4+a^2*b^10*c^4*d^4+a^3*b^8*c^5*d^4+a^4*b^6*
      c^6*d^4+a^5*b^4*c^7*d^4+a^6*b^2*c^8*d^4+a^7*c^9*d^4+a*b^13*c*d^5+a^2*b^11
      *c^2*d^5+a^3*b^9*c^3*d^5+a^4*b^7*c^4*d^5+a^5*b^5*c^5*d^5+a^6*b^3*c^6*d^5+
      a^7*b*c^7*d^5+a^2*b^12*d^6+a^3*b^10*c*d^6+a^4*b^8*c^2*d^6+a^5*b^6*c^3*d^6
      +a^6*b^4*c^4*d^6+a^7*b^2*c^5*d^6+a^8*c^6*d^6+a^4*b^9*d^7+a^5*b^7*c*d^7+a^
      6*b^5*c^2*d^7+a^7*b^3*c^3*d^7+a^8*b*c^4*d^7+a^6*b^6*d^8+a^7*b^4*c*d^8+a^8
      *b^2*c^2*d^8+a^9*c^3*d^8+a^8*b^3*d^9+a^9*b*c*d^9+a^10*d^10
\end{verbatim}
Therefore, we use a different strategy if either $(p-1)K_R$ is not Cartier or more generally if $R$ is $\bQ$-Gorenstein of index divisible by $p$ which appears to typically be faster.  We rely on the following observation, see \cite{}.
\[
\tau(\omega_R, K_R) \cong \tau(R).
\]
In fact, by embedding $\omega_R \subseteq R$, we can compute $g$ so that $\tau(\omega_R, K_R) = g \tau(R)$.  We can therefore find $\tau(R)$ if we can find $\tau(\omega_R, K_R)$.    
Next, if $K_R$ is $\bQ$-Cartier with $nK_R = \Div_R(f)$ for some $f \in R$ and $n > 0$, then 
\[
\tau(\omega_R, K_R) =\tau(\omega_R, f^{1/n}).
\]
Thus we directly compute $\tau(\omega_R, f^{1/n})$ via the command {\tt testModule(1/n, f)}.  Consider the following example, a $\mu_3$-quotient.  
\begin{verbatim}
i2 : T = ZZ/3[x,y];
i3 : S = ZZ/3[a,b,c,d];
i4 : f = map(T, S, {x^3, x^2*y, x*y^2, y^3});
i5 : I = ker f;
i6 : R = S/I;
i7 : testIdeal(R)
o7 = ideal 1
o7 : Ideal of R
\end{verbatim}
Which uses the logic described above.  

\begin{remark}[Non graded caveats]
It frequently can happen that $(I^{[p^e]} : I)/I^{[p^e]}$ is principal but Macaulay2 cannot identify the principal generator (since Macaulay2 cannot always find minimal generators of non-graded ideals or modules).  The same thing can happen when computing the $u$ corresponding to the map $T : \omega_{R^{1/p}} \to \omega_R$, as described in \autoref{} above.  In such situations, instead of a single $u$, we Macaulay2 will produce $u_1, \dots, u_n$ (all multiples of $u$, and $u$ is a linear combination of the $u_i$).  Instead of computing things like
\[
(u \cdot J)^{[1/p]},
\]
we will instead compute 
\[
(u_1 \cdot J)^{[1/p]} + \dots + (u_n \cdot J)^{[1/p]}
\]
which will produce the same answer.  
\end{remark}

We can similarly use the {\tt testIdeal} command to compute test ideals of pairs $\tau(R, f^t)$, and even more general mixed test ideals $\tau(R, f_1^{t_1} \cdots f_n^{t_n})$.  

\subsection{HSLG module, computing $F$-pure submodules of rank-1 Cartier modules}

Again consider the maps $T^e : \omega_{R^{1/p^e}} \to \omega_R$ we have considered throughout this section.  It is a theorem of Hartshorne-Speiser, Lyubeznik, and Gabber \cite{} that the descending images
\[
\omega_R \supseteq T(\omega_{R^{1/p}}) \supseteq \dots \supseteq T^e(\omega_{R^{1/p^e}}) \supseteq T_{e+1}(\omega_{R^{1/p^{e+1}}}) \supseteq \dots
\]
stabilize for $e \gg 0$.  


\section{Ideals compatible with given $p^{-e}$-linear map}

Throughout this section let $R$ denote a polynomial $\mathbb{K}[x_1, \dots, x_n]$. In this section we address the following question:
given a $p^{-e}$ linear map $\phi: R \rightarrow R$, what are all ideals $I\subseteq R$ such that $\phi(I)=I$?
We call these ideals $\phi$-compatible.

Recall that in section \ref{Section: p-linear maps} we identified $p^{-e}$ linear maps
with elements of $\Hom_R(F_*^e R, R)$ and it turns out that this is a principal $F_*^e R$-module generated
the \emph{trace map} $\pi\in \Hom_R(F_*^e R, R)$, constructed as follows.  {\hfill\large\color{red} [Add references]}\\
Let $B$ be a $\mathbb{K}^{p^e}$-basis for $\mathbb{K}$;
$F_*^e R$ is a free $R$-module with basis
$$\left\{ F_*^e b x_1^{\alpha_1} \dots x_n^{\alpha_n} \,|\, 0\leq \alpha_1, \dots, \alpha_n < p^e \right\}$$
and the trace map $\pi$ is the projection onto the free summand
$R F_*^e  x_1^{p^e-1} \dots x_n^{p^e-1}\cong R$.

We can now write our given $\phi$ as multiplication by some $F_*^e u$ followed by $\pi$ and it is not hard to see that
an ideal $I\subseteq R$ is $\phi$-compatible if and only if $u I \subseteq I^{[p^e]}$.

\begin{theorem}[ {\large\color{red} [Add references]}]
If $\phi$ is surjective, there are finitely many $\phi$-compatible ideals, consisting of all possible intersections
of $\phi$-compatible prime ideals.

In general, there are finitely many $\phi$-compatible prime ideals not containing $\sqrt{\Image \phi}$.

\end{theorem}

The method \emph{compatibleIdeals} produces the finite set of $\phi$-compatible prime ideals in the second statement of the theorem above.

\begin{verbatim}
i2 :      R = ZZ/3[u,v];
i3 :      u = u^2*v^2;
i4 :      compatibleIdeals(u)
o4 = {ideal v, ideal (v, u), ideal u}
o4 : List
\end{verbatim}

The defining condition $u I = I^{[p]}$ for $\phi$-compatible ideals allows us to
think of these in a dual form: write $\mathfrak{m}=(x_1, \dots, x_n)R$,
$E=E_{R\mathfrak{m}}(R_{\mathfrak{m}}/\mathfrak{m})=\HH^n_{\mathfrak{m}} (R)$, and let $T: E \rightarrow E$ be the $p^e$-linear map
induced from the Frobenius map on $R$.
If $\psi=u T$, then $\psi \Ann_E I \subseteq \Ann_E I$ if and only if $u I = I^{[p]}$!
Thus finding all $R$-submodules of $E$ compatible with $\psi=u T$ also amounts to finding all
$\phi=\pi \circ F_*^e u$ ideals.


Let $\mathbb{K}=\mathbb{Z}/2\mathbb{Z}$, $R=\mathbb{K}[x_1, x_2, x_3, x_4, x_5]$, $\mathfrak{m}=(x_1, \dots, x_5)R$, and let $I$ be the ideal of
$2\times 2$ minors of
$$
\left[
\begin{array}{c c c c}
 x_1 & x_2 & x_2 & x_5\\
 x_4 & x_4 & x_3 & x_1
\end{array}
\right]
$$
In (cf.~\cite[\S 9]{KatzmanParameterTestIdealOfCMRings}) it is shown that
there is a surjection $\Ann_E I \rightarrow \HH^2_{\mathfrak{m}} (R/I)$
which is compatible with with the induced $p^1$-linear map on $\HH^2_{\mathfrak{m}} (R/I)$
and the $p^1$-linear map $u T$ on $\Ann_E I$ with
$$u=x_1^3x_2x_3 + x_1^3x_2x_4+x_1^2x_3x_4x_5+ x_1x_2x_3x_4x_5+ x_1x_2x_4^2x_5+ x_2^2x_4^2x_5+x_3x_4^2x_5^2+ x_4^3x_5^2 .$$


\begin{verbatim}
i2 :      R=ZZ/2[x_1..x_5];
i3 :      I=minors(2, matrix {{x_1,x_2,x_2,x_5},{x_4,x_4,x_3,x_1}} )
                                                   2
o3 = ideal (x x  + x x , x x  + x x , x x  + x x , x  + x x , x x  + x x , x x
             1 4    2 4   1 3    2 4   2 3    2 4   1    4 5   1 2    4 5   1 2
     --------------------------------------------------------------------------
     + x x )
        3 5
o3 : Ideal of R
i4 :      u=x_1^3*x_2*x_3 + x_1^3*x_2*x_4+x_1^2*x_3*x_4*x_5+ x_1*x_2*x_3*x_4*x_5+
          x_1*x_2*x_4^2*x_5+ x_2^2*x_4^2*x_5+x_3*x_4^2*x_5^2+ x_4^3*x_5^2;
i5 :      compatibleIdeals(u)
             3        3        2                           2      2 2
o5 = {ideal(x x x  + x x x  + x x x x  + x x x x x  + x x x x  + x x x  +
             1 2 3    1 2 4    1 3 4 5    1 2 3 4 5    1 2 4 5    2 4 5
     --------------------------------------------------------------------------
        2 2    3 2                    2
     x x x  + x x ), ideal (x  + x , x  + x x ), ideal (x , x ), ideal (x , x ,
      3 4 5    4 5           1    2   1    4 5           4   1           4   1
     --------------------------------------------------------------------------

     x ), ideal (x , x , x , x ), ideal (x , x , x , x , x ), ideal (x , x ,
      5           5   4   1   3           5   4   3   2   1           5   4
     --------------------------------------------------------------------------

     x , x ), ideal (x , x , x ), ideal (x , x , x , x ), ideal (x , x , x ),
      2   1           4   1   3           4   3   2   1           4   2   1
     --------------------------------------------------------------------------
                               2
     ideal (x  + x , x  + x , x  + x x ), ideal (x  + x , x , x , x ), ideal
             3    4   1    2   2    4 5           3    4   2   1   5
     --------------------------------------------------------------------------
     (x , x , x )}
       2   1   5

o5 : List
\end{verbatim}

\section{Future plans}

\bibliographystyle{skalpha}
\bibliography{MainBib}



\end{document}
